% -----------------------------------------------------------------------------
% information
% -----------------------------------------------------------------------------

\def \docAuthor {Name:}
\def \docClass  {CPSC 120}
\def \docSchool {~}
\def \docTerm   {Spring 2014}
\def \docTitle  {Week 7---Truth Tables}


% -----------------------------------------------------------------------------
% document setup
% -----------------------------------------------------------------------------

\documentclass[12pt,letterpaper]{article}

\usepackage[includehead,
            includefoot,
            margin=1in,
            top=.25in,
            headheight=.75in,
            headsep=.25in,
            footskip=.25in,
           ]{geometry}

\usepackage[fleqn]{amsmath}
\usepackage{amssymb}
\usepackage{courier}
\usepackage{enumitem}
\usepackage{fancybox}
\usepackage{fancyhdr}
\usepackage{hyperref}
\usepackage{l3regex}
\usepackage{mathtools}
\usepackage{multicol}
\usepackage[normalem]{ulem}
\usepackage[x11names,table]{xcolor}

\usepackage{tikz}

\usepackage{listings}

\usepackage[doublespacing]{setspace}


% text ------------------------------------------------------------------------

\binoppenalty = 10000  % never break next to a binary operator
\relpenalty   = 10000  % never break next to a relation operator

\setlength{\parindent}{0em}
\setlength{\parskip}{1ex}

\setlist[itemize]{nosep,itemsep=.5ex,parsep=.5ex}

% math ------------------------------------------------------------------------

\setlength{\mathindent}{1cm}

% - "\begin{document}" resets these values, so they have to be treated
%   specially
\AtBeginDocument{
  \setlength{\abovedisplayskip}{1.5ex plus .5ex minus .5ex}
  \setlength{\belowdisplayskip}{1.5ex plus .5ex minus .5ex}
}

% source code -----------------------------------------------------------------

\lstset{
  basicstyle=\ttfamily,
  keywordstyle=\color{Firebrick3},
  commentstyle=\color{blue},
}

% header and footer -----------------------------------------------------------

\pagestyle{fancy}

\lhead{\docClass}
\rhead{\docTitle}
\cfoot{\thepage}
\renewcommand{\headrule}{\hrule height 0.4pt}
\renewcommand{\footrule}{\hrule height 0.4pt}

\fancypagestyle{firstpage}{
  \fancyhead[L]{\docAuthor\\\docClass}
  \fancyhead[C]{\docSchool\\}
  \fancyhead[R]{\docTerm\\\docTitle}
}


% -----------------------------------------------------------------------------
% macros
% -----------------------------------------------------------------------------

% abbreviations ---------------------------------------------------------------

\def \<{\langle}
\def \>{\rangle}

\def \ε{\varepisilon}
\def \θ{\vartheta}
\def \κ{\varkappa}
\def \π{\varpi}
\def \ρ{\varrho}
\def \σ{\varsigma}
\def \φ{\varphi}

\def \Γ{\varGamma}
\def \Δ{\varDelta}
\def \Θ{\varTheta}
\def \Λ{\varLambda}
\def \Ξ{\varXi}
\def \Π{\varPi}
\def \Σ{\varSigma}
\def \Υ{\varUpsilon}
\def \Φ{\varPhi}
\def \Ψ{\varPsi}
\def \Ω{\varOmega}

% special characters ----------------------------------------------------------

\catcode `α = \active \let α \alpha
\catcode `β = \active \let β \beta
\catcode `γ = \active \let γ \gamma
\catcode `δ = \active \let δ \delta
\catcode `ε = \active \let ε \epsilon
\catcode `ζ = \active \let ζ \zeta
\catcode `η = \active \let η \eta
\catcode `θ = \active \let θ \theta
\catcode `ι = \active \let ι \iota
\catcode `κ = \active \let κ \kappa
\catcode `λ = \active \let λ \lambda
\catcode `μ = \active \let μ \mu
\catcode `ν = \active \let ν \nu
\catcode `ξ = \active \let ξ \xi
\catcode `ο = \active \let ο o
\catcode `π = \active \let π \pi
\catcode `ρ = \active \let ρ \rho
\catcode `σ = \active \let σ \sigma
\catcode `τ = \active \let τ \tau
\catcode `υ = \active \let υ \upsilon
\catcode `φ = \active \let φ \phi
\catcode `χ = \active \let χ \chi
\catcode `ψ = \active \let ψ \psi
\catcode `ω = \active \let ω \omega

\catcode `Α = \active \let Α A
\catcode `Β = \active \let Β B
\catcode `Γ = \active \let Γ \Gamma
\catcode `Δ = \active \let Δ \Delta
\catcode `Ε = \active \let Ε E
\catcode `Ζ = \active \let Ζ Z
\catcode `Η = \active \let Η H
\catcode `Θ = \active \let Θ \Theta
\catcode `Ι = \active \let Ι I
\catcode `Κ = \active \let Κ K
\catcode `Λ = \active \let Λ \Lambda
\catcode `Μ = \active \let Μ M
\catcode `Ν = \active \let Ν N
\catcode `Ξ = \active \let Ξ \Xi
\catcode `Ο = \active \let Ο O
\catcode `Π = \active \let Π \Pi
\catcode `Ρ = \active \let Ρ P
\catcode `Σ = \active \let Σ \Sigma
\catcode `Τ = \active \let Τ T
\catcode `Υ = \active \let Υ \Upsilon
\catcode `Φ = \active \let Φ \Phi
\catcode `Χ = \active \let Χ X
\catcode `Ψ = \active \let Ψ \Psi
\catcode `Ω = \active \let Ω \Omega

% other -----------------------------------------------------------------------

% - sometimes a "\par", especially at the end of a block, is necessary to
%   prevent an extra (empty) paragraph from appearing in the output
% - `\color{.!50}` means 50 percent of the current color

     \def \note     #1{{\color{.!50}#1\par}}
\long\def \longnote #1{{\color{.!50}#1\par}}

\long\def \answer #1{
  \filbreak\par #1 \par
}

% so we can show or hide the numbers easily
% - now defined in the including files
% \def \0 {0}}
% \def \1 {1}}
% \def \0 {~}
% \def \1 {~}



% -----------------------------------------------------------------------------
% document
% -----------------------------------------------------------------------------

\begin{document}
\thispagestyle{firstpage}

\lstset{language=C}
\arrayrulecolor{gray}

\textbf{Notes:}
\begin{itemize}
  \item \textbf{C++ Standard: 4.7 Integral conversions: 4:}\\
    If the destination type is \verb!bool!, see 4.12. If the source type is
    \verb!bool!, the value \verb!false! is converted to zero and the value
    \verb!true! is converted to one.
  \item \textbf{C++ Standard: 4.12 Boolean conversions:}\\
    A prvalue of arithmetic, unscoped enumeration, pointer, or pointer to
    member type can be converted to a prvalue of type \verb!bool!. A zero
    value, null pointer value, or null member pointer value is converted to
    \verb!false!; any other value is converted to \verb!true!. For
    direct-initialization (8.5), a prvalue of type \verb!std::nullptr_t! can be
    converted to a prvalue of type \verb!bool!; the resulting value is
    \verb!false!.
\end{itemize}

\vspace{5em}

\begin{tabular}{|c|c|c|c|c|c|c|c|}
  \hline
  \verb:true: & \verb:false: & \verb:bool(-2): & \verb:bool(-1):
  & \verb:bool(0): & \verb:bool(1): & \verb:bool(2): & \verb:bool(3):\\
  \hline
  1 & 0 & \1 & \1 & \0 & \1 & \1 & \1 \\
  \hline
\end{tabular}

\vspace{1em}

\begin{tabular}{|c|c|c|c|c|c|}
  \hline
  \verb:bool('\0'): & \verb:bool('y'): & \verb:bool('n'): & \verb:bool("yes"):
  & \verb:bool("no"): & \verb:bool("hello"):\\
  \hline
  \0 & \1 & \1 & \1 & \1 & \1 \\
  \hline
\end{tabular}

\vspace{1em}

\begin{tabular}{|c|c|c|c|}
  \hline
  \verb:bool(-5.7): & \verb:bool(0.0): & \verb:bool(5.7): & \verb:bool(42.8):\\
  \hline
  \1 & \0 & \1 & \1 \\
  \hline
\end{tabular}

\vspace{1em}

\begin{tabular}{|c|c|c|c|c|c|}
  \hline
  \verb:a: & \verb:b: & \verb:a == b: & \verb:a != b: & \verb:a && b:
  & \verb:a || b:\\
  \hline 0 & 0 & \1 & \0 & \0 & \0 \\
  \hline 0 & 1 & \0 & \1 & \0 & \1 \\
  \hline 1 & 0 & \0 & \1 & \0 & \1 \\
  \hline 1 & 1 & \1 & \0 & \1 & \1 \\
  \hline
\end{tabular}

\vspace{1em}

\begin{tabular}{|c|c|c|c|c|c|}
  \hline
  \verb:a: & \verb:b: & \verb:a == b && 'y': & \verb:a == b || 'y':
  & \verb:a != b && 'y': & \verb:a != b || 'y':\\
  \hline 0 & 0 & \1 & \1 & \0 & \1 \\
  \hline 0 & 1 & \0 & \1 & \1 & \1 \\
  \hline 1 & 0 & \0 & \1 & \1 & \1 \\
  \hline 1 & 1 & \1 & \1 & \0 & \1 \\
  \hline
\end{tabular}

\vspace{1em}

\begin{tabular}{|c|c|c|c|c|c|c|}
  \hline
  \verb:a: & \verb:b: & \verb:c: & \verb:a==b && a==c:
  & \verb:a==b || a==c: & \verb:a!=b && a!=c:
  & \verb:a!=b || a!=c:\\
  \hline 0 & 0 & 0 & \1 & \1 & \0 & \0 \\
  \hline 0 & 0 & 1 & \0 & \1 & \0 & \1 \\
  \hline 0 & 1 & 0 & \0 & \1 & \0 & \1 \\
  \hline 0 & 1 & 1 & \0 & \0 & \1 & \1 \\
  \hline 1 & 0 & 0 & \0 & \0 & \1 & \1 \\
  \hline 1 & 0 & 1 & \0 & \1 & \0 & \1 \\
  \hline 1 & 1 & 0 & \0 & \1 & \0 & \1 \\
  \hline 1 & 1 & 1 & \1 & \1 & \0 & \0 \\
  \hline
\end{tabular}

\end{document}

